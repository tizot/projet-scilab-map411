\section{Position du problème}

\begin{enumerate}[label=\fbox{\textbf{Q\arabic*}}, series=questions]
\item L'intensité d'un point matériel est supposée constante le long de son déplacement. On en déduit:
\begin{align*}
  \dfrac{\D \tilde{I}}{\D t} &= \dfrac{\D x}{\D t} \partial_x I + \dfrac{\D y}{\D t} \partial_y I + \partial_t I \\
                             &= (\partial_x I, \partial_y I) \cdot (u, v) + \partial_t I \\
                             &= 0 \quad\text{(par hypothèse)}
\end{align*}
On en déduit l'équation:
\begin{equation}
  \label{eq:general}
  \nabla I \cdot h + \partial_t I = 0 \quad \text{sur } \Omega \times \interff{0}{T}
\end{equation}
L'inconnue de cette équation est le vecteur $h = (u,v)$, correspondant à la vitesse d'un point matériel suivi, de coordonnées initiales $(x_0, y_0)$. 

On ne peut pas résoudre cette équation sans hypothèse supplémentaire sur $h$. En effet, fondamentalement, l'information sur $h$ nous provient de la connaissance de l'intensité $I$, or cette grandeur est \og{}de \emph{dimension 1}\fg{} alors que $h$ a \emph{deux} composantes indépendantes.
\end{enumerate}